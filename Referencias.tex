\documentclass{article}


\begin{document}

\title{Referencias en \LaTeX}
\author{Manuel Olmos Antillón}
\date{\today}

\maketitle

\section{Introduction}
This is a simple introduction to LaTeX. LaTeX is a high-quality typesetting system; it includes features designed for the production of technical and scientific documentation.

\section{Main Content}
Here you can write the main content of your document. You can include equations, figures, tables, and more.

\section{Conclusion}
This is the conclusion of the document. LaTeX makes it easy to create professional-looking documents.

\subsection{Sub-Tema1}
Este es un sub-tema. Entonces la referencia a \cite{Enr13} es muy importante.

\begin{thebibliography}{99} % 99 por que no se espera que haya más de 99 referencias

\bibitem{Enr13}
M. Enríquez, \textit{Introducción a \LaTeX}, 2013.

\bibitem{Lam94}
L. Lamport, \textit{\LaTeX: A Document Preparation System}, 2nd Edition, 1994.

\end{thebibliography}

\end{document}