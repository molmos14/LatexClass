\documentclass{article}
\usepackage{authblk}
\usepackage{amsmath}
\usepackage{amssymb}

\title{Actividad 4 Escritura de ecuaciones}
\author{Manuel Olmos Antillón}
\date{\today}
\affil[1]{Ingeniería en Sistemas Computacionales, Escuela de Ingeniería y Ciencias, Tecnológico de Monterrey}

\begin{document}

\maketitle

\section{Propósito}
Usar código Latex para la escritura de ecuaciones.

\section{Descripción}
De forma individual investiga lo necesario y realiza las siguientes ecuaciones en Latex:
\\

\noindent Primera ecuación:
\[
\alpha_{\text{in}} = \sqrt{\beta^2 + \rho^2}
\]
\\

\noindent Segunda ecuación:
\begin{equation} \label{eq:ecuacion1}
\frac{\partial u}{\partial t} = f \left( \frac{\partial ^2}{\partial x^2}, u\right)
\end{equation}

\noindent Tercera ecuación:
\begin{equation} \label{eq:ecuacion2}
\int_{0}^{\infty}  \,f(x) \,dx
\end{equation}
\\

\noindent Cuarta ecuación:
$
\mathrm{N}_{\text{2(g)}} + 3\mathrm{H}_{\text{2(g)}} \rightarrow 2\mathrm{NH}_{\text{3(g)}}
\\
$

\noindent Quinta ecuación:
$
\omega \in \mathbb{R} \subset \mathbb{C} 
$
\\

\noindent Las ecuaciones (\ref{eq:ecuacion1}), (\ref{eq:ecuacion2}) son ejemplos de ecuaciones que se pueden escribir en Latex.

\end{document}