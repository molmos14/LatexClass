\documentclass{article}
\usepackage{array}

\begin{document}

Este es un documento de ejemplo que muestra diferentes formas de escribir divisiones en \LaTeX.
\\

% Este es un ejemplo usando notación \[ \dots \]:
\textbf{notación con \texttt{\textbackslash[ \dots \textbackslash]}:}
\[ \frac{1}{2} \]

% Primero, usando la notación \( \dots \):
\textbf{notación con \texttt{\textbackslash( \dots \textbackslash)}:}
\( \frac{a}{b} \)
\\

% Segundo, usando la notación $ \dots $:
\textbf{notación con \texttt{\$ \dots \$}:}
$ \frac{c}{d} $
\\

% Tercero, usando el entorno \texttt{math}:
\textbf{notación con \texttt{math}:}
\begin{math}
\frac{e}{f}
\end{math}
\\

\noindent Para cuando te piden que no exista identación en el párrafo, se puede usar \texttt{\textbackslash noindent}:
\\

No se recomienda colocar estos espacios \texttt{\textbackslash hspace\{1cm\}} en el código, ya que se pueden perder en la compilación.
Al igual que con \texttt{\textbackslash vspace\{1cm\}}.
\\ \\

\textbf{Ejemplos de fracciones}

En el centro sin numerar, no se puede referenciar:
    \[\frac{1}{2} = 2^{-1}\]
    \begin{displaymath}
        \left( \frac{1}{2} \right) = 2^{-1}
        % Este es un ejemplo de cómo se puede escribir una fracción en el centro de la página. con parentesis
    \end{displaymath}

Ecuación centrada y Numerada, se puede referenciar:
{\Huge
\begin{equation}
    \frac{1}{2} = 2^{-1}
    \label{eq:fraccion} % para que sirve eq:fraccion ? Es para referenciar la ecuación
\end{equation}}
\\ \\

Texto encuación (\ref{eq:fraccion}) centrada y numerada: 
    $\partial{f(x)}$ texto texto texto \dots
\\

\noindent Numeros raros ejemplos:
$\int_{1}^{1}  \,dx$,
$\sum_{n = 1}^{\infty}$,
$\prod_{n = 1}^{\infty}$,
$\lim_{x \to 0}$.
\\ \\

\noindent Un ejemplo de una matriz de ecuaciones alineada sería usando array:
\[
\begin{array}{c}
    x + y = 1 \\
    x - y = 2
\end{array}
\]
\\
Otro ejemplo:

\begin{eqnarray}
    y &= x^2 + 2x + 1 \\
    y &= (x + 1)^2 \nonumber
\end{eqnarray}

\begin{eqnarray}
    y &=& x^2 + 2x + 1 \\
    y &=& (x + 1)^2 \nonumber
\end{eqnarray}
% para que sirve el &=& ? Es para alinear las ecuaciones

\end{document}